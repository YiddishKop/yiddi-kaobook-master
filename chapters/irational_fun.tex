%\setchapterstyle{kao}
\setchapterpreamble[u]{\margintoc}
\chapter{第四讲:反比例函数---初识}
%\labch{pingxing}


\section{反比例函数的定义}
反比例函数是相对正比例函数而言的,正比例函数的图像是一条直线,而反比例函数的图像是一条曲线,这条曲线叫做双曲线。双曲线是一种非常重要的曲线。

正比例函数描述的是一种“同盟关系”---“你增加我就增加,你减少我也减少”;而反比例关系描述的则是一种“敌对关系”---“你增加我就减少,你减少我则增加”!

正比例函数与反比例函数的形式也是对应的,保证 $k$ 是常数,我们给出如下表达式:

正比例函数
$$y=k\cdot x, \hspace{1cm}k\neq 0$$

反比例函数
\marginnote{ 
    反比例函数成立的条件:
\begin{itemize}    
\item $k\neq 0$
\item x 的系数必须是-1
\end{itemize}
}

$$y=\frac{k}{x}, \hspace{1cm}k\neq 0$$

我们对反比例函数进行一下变形可以得到反比例函数的变形\sidenote{反比例函数的变形形式是 $\fbox{y\cdot x=k}$,这种形式更加直观,更容易理解。}
$$y\cdot x=k, \hspace{1cm}k\neq 0$$


$x$ 与 $y$ 这两个变量在这个式子中展现的是一种敌对的争抢关系,因为 $k$ 是固定的。这种关系
在生活中也是非常常见的.




\example{
    下列关于x的函数中:\marginnote{注意 k 非零}
    $①y=2/x  \ \ \  ②y=(-4)/3x \ \ \ ③y=k/x \ \ \  ④y=(m^2+2)/x \ \ \ $中
一定是反比例函数的有( \ \   )\\
A.1个	\hspace{1cm}	B. 2个	 \hspace{1cm}	C. 3个	\hspace{1cm}	D. 4个}

\begin{answers}
    【解析】显然①、②是反比例函数,
    ③虽然具备反比例函数的形式,但是不满足反比例函数的要求,
    也就是不能保证比例常数k是一个“非零”常数. ④具备反比例函数的形式,
    同时也能保证比例常数$m^2+2≠0$,这是因为m是实数,所以$m^2≥0$,$m^2+2>0$,
因此①、②、④是反比例函数.
【答案】C
\end{answers}

\begin{example}
    已知y与$x^2$成反比例,当x=3时,y=4,则y是x的( \ \ \   )\\
    \marginnote{注意 x 系数为 -1}
A. 正比例函数 \ \ \  B.一次函数 \ \ \    C.反比例函数 \ \ \   D.以上都不是
\end{example}

\begin{answers}
    D, 系数不为-1
\end{answers}


\example{已知$y=(m^2+2m) x^{m^2+m-1}$是关于x的反比例函数,求m的值及函数的解析式。}

\begin{answers}
【解析】根据反比例函数$y=k/x (k≠0)$也可以写成 $y=kx^(-1) (k≠0)$,可知:
        $m^2+m-1=-1①$
        $m^2+2m≠0②$
        由①得:m=0或m=-1;由②得:m≠0且m≠2,
∴m=-1,则y=-1/x
【答案】m=-1,则y=-1/x
\end{answers}











\section{生活中的反比例函数}

\subsection{速度与时间成反比}

描述: 在一段固定的距离内,行驶速度与所需时间成反比例关系。\\
案例: 从徐州到上海,车速提高一倍,那么所需时间就会减少一半。\\
公式: $\fbox{v\cdot t = k}$,其中 $v$ 为速度,$t$ 为时间,$k$ 为常数。

\subsection{力臂与力成反比}

描述: 杠杆平衡时,动力臂与阻力臂成反比例关系。\\
案例: 使用撬棍撬动物体时,动力臂越长,所需的动力就越小。\\
公式: $\fbox{F\cdot l = k}$,其中 $F$ 为力,$l$ 为力臂,$k$ 为常数。

\subsection{浓度与体积成反比}

描述: 在溶质质量不变的情况下,溶液的浓度与体积成反比例关系。\\
案例: 将一定量的糖溶解在水中,增加水的体积,糖水的浓度就会降低。\\
公式: $\fbox{C\cdot V = k}$,其中 $C$ 为浓度,$V$ 为体积,$k$ 为常数。\\


\begin{example}
写出下列各题中所要求的两个相关量之间的函数关系式,并指出函数的类别.\\
(1)商场推出分期付款购电脑活动,每台电脑12000元,首付4000元,以后每月付y元,x个月全部付
清,则y与x的关系式为$\uds\uds$,是$\uds$函数.\\

(2)某种灯的使用寿命为1000小时,它的使用天数y与平均每天使用的小时数x之间的
关系式为$\uds\uds$,是$\uds$函数.\\

(3)设三角形的底边、对应高、面积分别为a、h、S.\\
当a=10时,S与h的关系式为$\uds\uds$,是$\uds$函数;\\
当S=18时,a与h的关系式为$\uds\uds$,是$\uds$函数.\\

(4)某工人承包运输粮食的总数是 W 吨,每天运 x 吨,共运了 y 天,则y与x的关系
式为$\uds\uds$,是$\uds$函数.
\end{example}

\begin{answers}
    【答案】(1)y=8000/x,反比例;
(2)y=1000/x,反比例;
(3)S=5h,正比例,a=36/h,反比例;
(4)y=W/x,反比例.

\end{answers}





\subsection{长方形的长与宽成反比}
最后我们考虑长方形。长方形的面积是长与宽的乘积,如果面积固定,长与宽成反比例关系。\\
公式: $\fbox{l\cdot w = k}$,其中 $l$ 为长,$w$ 为宽,$k$ 为常数。

【注意】可以看出这里的 k 表示的就是长方形的面积,\ubi{"k就是面积"} 将成为我们探究反比例函
数的\ubi{金钥匙}。这一点会在之后的教学中进一步说明。但是现在请大家记住:
\begin{itemize}
    \item \ubi{面积}对于\ubi{反比例函数}非常重要!
    \item 对 k 的讨论等效于讨论\ubi{长方形的面积}
\end{itemize}


\begin{example}
现在假设长方形的面积始终是8,以x轴的坐标作为长,请在同一个坐标系内绘制长为1到8的整数长度的长方形图形。\\
\includegraphics{4-1-1.png}
\end{example}


\section{反比例函数的基本图像}

\subsection{k>0的情况}
如果你画的图形是对的话,他大概的样子应该如下图所示。

\includegraphics[width=0.8\textwidth]{4-1-2.png}

现在请你\ubi{用你认为最合适的线(曲线或者直线)将这些点连接起来}

他大概应该是这样的:
\marginnote{
    请思考:
    \begin{itemize}
        \item 回忆一下这个图形的作图过程,他代表的是什么?请尝试从图像上每一个点的意义来解释。\\
        \includegraphics{4-1-5.png}
        \uds\uds\uds\uds\\
        \uds\uds\uds\uds\\
        \uds\uds\uds\uds\\
        \item 如果长方形的面积是 4 或者 16(k=4, k=16),那么这个双曲线会是什么样子,你能猜出来么?\\
        \uds\uds\uds\uds\\
        \uds\uds\uds\uds\\
        \uds\uds\uds\uds\\
    \end{itemize}
}

\includegraphics[width=0.8\textwidth]{4-1-3.png}


\cleardoublepage
\definition{反比例函数的图像是一种\ubi{曲线},这种曲线包含两个分支,我们叫它\ubi{双曲线}。}
\marginnote{就像永远把一次函数想象成不同倾斜程度的直线一样,反比例函数永远带着“\ubi{面积}”的想法在大脑中绘图,这是他的基因!}

\subsection{k<0的情况}

同理当 k=-8 时,请再次画出反比例函数的图像,注意先画点再/描线!\\

\marginnote{
    我们分析函数的性质主要包括以下三个方面
    \begin{itemize}
        \item 对称性:中心对称和轴对称图形的函数表达式是什么样子
        \item 增减性:一个函数在某个区间内是增还是减
        \item 周期性:(超纲)高中学习
    \end{itemize}
    现在反比例函数的图像你已经学会绘制了,请你就1、2两个方面进行分析。
}

\includegraphics[width=0.8\textwidth]{4-1-1.png}


请绘制完图像后对比下图:\\

\includegraphics{4-1-7.png}

\begin{theorem}
    我们能得出的结论是:
    \begin{itemize}
        \item 当 k>0 时,图像经过第一、三象限。
        \item 当 k<0 时,图像经过第二、四象限。
    \end{itemize}
\end{theorem}

\begin{example}
    反比例函数$y=\frac{-1}{x}$的图象大致是图中的(    ).\\
    \includegraphics{4-exp-1.png}
\end{example}

\begin{answers}
    【答案】D
\end{answers}

\begin{example}
    在下图中,反比例函数$y=\frac{k^2+1}{x}$的图像大致是(    )\\
    \includegraphics{4-exp-2.png}
\end{example}

\begin{answers}
    D
\end{answers}


\begin{example}
    已知点P (1,a)在反比例函数 $y=\frac{k}{x}$ (k≠0)的图像上,其中$a=m^2+2m+3$ (m为实数),则这个函数的图像在第$\uds$象限.
\end{example}

\begin{answers}
    【答案】一、三象限.
\end{answers}

\begin{example}
    如果点(-t,-2t)在双曲线$y=\frac{k}{x}$上,那么k$\uds$0,双曲线在第$\uds$象限.
\end{example}


\begin{answers}
    【答案】>;一、三.
\end{answers}

\begin{example}
    已知$y=(a-1)x^a$是反比例函数,则它的图象在(    ).\\
A.第一、三象限	\ \ \ \ 	B.第二、四象限 \ \ \ \\
C.第一、二象限	\ \ \ \ 	D.第三、四象限 \ \ \
\end{example}

\begin{answers}
    【答案】B
\end{answers}


\example{在反比例函数$y=\frac{k-5}{x}$图象的每一支曲线上,y都随x的增大而减小,则k的取值范围是 ( \ \  )\\
A.k>5     \hspace{1cm}     B.k>0   \hspace{1cm}   C.k<5    \hspace{1cm}     D.k<0}

\begin{answers}
    【解析】由反比例函数y都随x的增大而减小可以判断图象分
    布在第一、三象限,因此比例系数k-5的符号
    是正数,解不等式k-5>0.即k>5
\end{answers}


\cleardoublepage
\section{反比例函数作业1}

\begin{exercise}    
    已知反比例函数$y=k/x$的图像在第二、第四象限内,函数图像上有两
    点$A(2√7,y_1 )$,$B(5,y_2 )$,则$y_1$与$y_2$的大小关系为( \ \ )\\
A.$y_1>y_2 $   \hspace{1cm}     B. $y_1=y_2$    \hspace{1cm}     C. $y_1<y_2$   \hspace{1cm}      D. 无法确定
\end{exercise}

\begin{answers}
    由题意可得$A(2√7,y_1 ),B(5,y_2 )$均在第四象限内,
    \because 反比例函数的图像在第二、第四象限内, \therefore k<0,
    在每个象限内,y随x的增大而增大,
    又2√7>5.\therefore $y_1>y_2$.
    【答案】A.
\end{answers}


\begin{exercise}
    若点$A(-1,y_1)$、$B (2,y_2)$、$C (π,y_3)$都是反比例函数$y=(k^2+1)/x$的图像上,试比较$y_1$、$y_2$、$y_3$的大小关系\uds\uds.
\end{exercise}


\begin{answers}
    $k^2+1>0$,所以$y=(k^2+1)/x$的函数在一、三象限内,易得$y_2>y_3>y_1$.
    【答案】$y_2>y_3>y_1$.
    
\end{answers}


\begin{exercise}
    反比例函数$y=(2m-1)x^{m^2-2}$,当x>0时,y随x的增大而增大,则m的值是(    ).\\
A.±1 \hspace{1cm}	B.小于1/2的实数 \hspace{1cm}	C.-1 \hspace{1cm}	D.1
\end{exercise}

\begin{answers}
    【解析】根据题意得,$m^2-2=-1$且2m-1<0,∴m=-1
【答案】C
\end{answers}


\begin{exercise}
    在同一坐标系中,$y=(m-1)x$与$y=-m/x$的图象的大致位置不可能的是(    ).\\
    \includegraphics{4-exp-3.png}
\end{exercise}

\begin{answers}
    【答案】A
\end{answers}

\begin{example}
    函数$y=ax-a$与$y=a/x, (a≠0)$在同一直角坐标系中的图象可能是(   )\\
    \includegraphics{4-exp-4.png}
\end{example}

\begin{answers}
    D
\end{answers}

\begin{exercise}
    在对物体做功一定的情况下,力F (牛)与此物体在力的方向上移动的距离s (米)成反比例函数关
    系,其图像如图所示,P(5,1)在图像上,则当力达到10牛时,物体在力的方向上移动的距离是$\uds$米.\\
    \includegraphics[width=0.3\textwidth]{4-exp-6.png}
\end{exercise}

\begin{answers}
    设反比例函数关系式为F=k/s,
将点P(5,1)代入,求得k=5.
\therefore 此函数关系式为F=5/x.
当F=10时,x=0.5米.
【答案】 0.5米
\end{answers}