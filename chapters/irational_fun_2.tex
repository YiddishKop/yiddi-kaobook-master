%\setchapterstyle{kao}
\setchapterpreamble[u]{\margintoc}
\chapter{第五讲:反比例函数图像的深入探讨}
%\labch{pingxing}


\section{不同 k 值对图像的影响}

\subsection{k 与矩形面积}

当 k 取不同于 8 的值的时候,反比例函数图像会发生什么变化呢?

\fbox{思考过程}
\begin{itemize}
    \item 首先,双曲线的本质不会产生变化,依旧是曲线。
    \item 其次,k 的几何意义是“面积”,所以从面积的角度考虑曲线会产生什么改变。
\end{itemize}

\marginnote{绘制完成后你发现了什么?加深对 k 的理解对于反比例函数非常重要!}
\begin{example}
    请你在同一个坐标系内画出 k=4 和 k=16 的反比例函数图像,并且比较他们之间的差异。(图中已经帮助你绘制好了 k=8 的图像):\\
    \includegraphics{4-1-4.png}
\end{example}



\cleardoublepage
\begin{theorem}
    因为 $y\cdot x=k$ 中的k代表的就是面积,所以表达式对应的
曲线就是\ubi{面积为k的所有点形成的线},当面积不同,也就是 k 值不同时,形成的曲线自然不同,
最直观的表现就是曲线是更“胖”还是更“瘦”
\sidenote{再进一步,我们还可以用其与原点的距离来进
一步理解,后面会结合反比例函数的对称性
展开这部分内容.}。
\end{theorem}

\vspace{0.5cm}


\hspace{1cm} $y=\frac{4}{x}$ \hspace{3cm} $y=\frac{8}{x}$ \hspace{2cm} $y=\frac{16}{x}$\\
\includegraphics{4-1-6.png}\\

现在我们将 $y=\frac{4}{x}$ 、$y=\frac{8}{x}$、 $y=\frac{16}{x}$ 的图像放在一起,请你标出各自曲线对应的表达式?\\


\includegraphics{5-1-1.png}\\


了解 k 的几何意义是面积之后,我们很容易可以得到一些好用的结论,这对我们加强对反比例函数的理解非常重要,
更复杂的结论我们会在后面的专题部分讲解,考虑到大家的接受程度,这里只介绍两个简单结论。


\cleardoublepage

\subsection{关于 k 的两个简单结论}

\begin{theorem}
    双曲线上任一点到坐标轴作垂线,垂线与坐标轴形成的矩形面积为 $|k|$;与原点围成的三角形面积为 $\frac{|k|}{2}$
    \begin{marginfigure}
        \includegraphics{5-1-2.png}
        \includegraphics{5-1-3.png}
    \end{marginfigure}
\end{theorem}

\begin{theorem}
    当 k 值增大时,双曲线会“放开”;当 k 值减小时,双曲线会“缩小”。
\end{theorem}


\begin{example}
    如图,点P在反比例函数的图像上,过P点作PA⊥x轴于A点,作PB⊥y
    轴于B点,矩形OAPB的面积为9,则该反比例函数的解析式为$\uds\uds$\\
    \includegraphics[width=0.3\textwidth]{5-exp-1.png}                
\end{example}

\begin{answers}
    $y=\frac{9}{x}$
\end{answers}


\begin{example}
    反比例函数$y=k/x$的图像如图所示,点M是该函数图像上一点,MN垂直于x轴,垂足是
    点N,如果$S_{ΔMON}=2$,则k的值为( \ \ \ )\\
    A. 2   \hspace{2cm}     B. -2    \hspace{2cm}    C. 4    \hspace{2cm}       D. -4\\
    \includegraphics[width=0.3\textwidth]{5-exp-2.png}
\end{example}

\begin{answers}
    答案D
\end{answers}


\begin{example}
    如图,在Rt⁡ΔAOB中,点A是直线$y=x+m$与双曲线$y=m/x$在第一象限的交点,且$S_{ΔAOB}=2$,则m的值是\uds\uds.
    \includegraphics[width=0.5\textwidth]{5-exp-3.png}
\end{example}


\begin{answers}
    【答案】已知$S_ΔAOB=2$. 因为 |m|/2=2, 所以 m>0, 所以 m=4.
\end{answers}


\cleardoublepage
\begin{example}
    如图,在函数 $y=\frac{2}{x}$ 的图像上任取一点A,过点A作y轴的垂线交
    函数 $y=-\frac{8}{x}$ 的图像于点B,连接OA,OB,则▲AOB的面积是( \ \ )\\
    A.3 \hspace{2cm}	B.5 \hspace{2cm}	C.6 \hspace{2cm}	D.10\\
    \includegraphics[width=0.7\textwidth]{5-exp-5.png}
\end{example}

\begin{answers}
    答案B
\end{answers}

\vspace{2cm}

\begin{example}
    \marginnote{【思考】方程的思想无处不在,涉及到函数问题的时候,主要是通过等量关系去建立方程,
    而建立方程的前提是设坐标。}
    已知点(1,3)在函数$y=k/x (x>0)$的图像上,矩形ABCD的边BC在x轴上,E是对角线BD的中点,函
    数$ y=k/x (x>0)$的图像经过A、E两点,若∠ABD=45°,求E点的坐标.\\
    \includegraphics[width=0.5\textwidth]{5-exp-4.png}
\end{example}

\begin{answers}
    点(1,3)在函数y=k/x的图像上,k=3.\\
又E也在函数y=k/x的图像上,故设E点的坐标为(m,3/m).\\
过E点作EF⊥x轴于F,则EF=3/m.\\
又E是对角线BD的中点,AB=CD=2EF=6/m.\\
故A点的纵坐标为6/m,代入y=3/x中,得A点坐标为 (m/2,6/m).\\
因此BF=OF-OB=m-m/2=m/2.由∠ABD=45°,得∠EBF=45°,BF=EF.\\
即有m/2=3/m.解得m=±√6.而m>0,故m=√6.则E点坐标为 (√6,√6/2).\\
\end{answers}


\cleardoublepage
\section{反比例函数的对称性与增减性}

\subsection{反比例函数的对称性}

通过八上的轴对称以及八下的中心对称,我们加强了对图形对称性的理解,函数自带图像,所以我们
也可以通过函数的对称性来加深对函数的理解。

\begin{theorem}
    反比例函数既是轴对称图形也是中心对称图形,它同时具有两个对称轴和
    一个对称中心,对于 $y=\frac{k}{x}$ 来说:
    \begin{itemize}
        \item 对称轴:y=x 和 y=-x
        \item 对称中心:原点(0,0)
    \end{itemize}
\end{theorem}

\includegraphics[width=0.4\textwidth]{5-1-5.png}
\includegraphics[width=0.4\textwidth]{5-1-6.png}
\includegraphics[width=0.4\textwidth]{5-1-7.png}


\subsection{反比例函数的增减性}
\begin{theorem}
    反比例函数的图像是双曲线,所有具有两个分支,这两个分支保持相同的增减性。
    \begin{itemize}
        \item 当 k>0 时,反比例函数是递减的
        \item 当 k<0 时,反比例函数是递增的
    \end{itemize}
\end{theorem}

\includegraphics[width=0.5\textwidth]{5-1-9.png}  \hspace{1cm}
\includegraphics[width=0.5\textwidth]{5-1-8.png}

\section{反比例函数图像相关练习}

\begin{example}
    如图是三个反比例函数$y=k_1/x$、$y=k_2/x$、$y=k_3/x$在x轴上方的图象,由此观
    察得到$k_1$、$k_2$、$k_3$的大小关系为\uds.
    \includegraphics[width=0.5\textwidth]{5-exp-6.png}
\end{example}

\begin{answers}
    $k_1<k_2<k_3$
\end{answers}

\begin{example}
    在反比例函数$y=(k-5)/x$图象的每一支曲线上,y都随x的增大而减小,则k的取值范围是 ( \ \  )\\
A.k>5      \hspace{1cm}    B.k>0   \hspace{1cm}    C.k<5    \hspace{1cm}      D.k<0
\end{example}

\begin{answers}
    【答案】A
\end{answers}

\vspace{1cm}

\begin{example}
    已知反比例函数$y=k/x$的图像在第二、第四象限内,函数图像上有两点$A(2√7,y_1)$,$B(5,y_2)$,则$y_1$与$y_2$的大小关系为( \  \ )\\
A.$y_1>y_2$     \hspace{1cm}    B. $y_1=y_2$   \hspace{1cm}     C. $y_1<y_2$   \hspace{1cm}     D. 无法确定
\end{example}

\begin{answers}
    由题意可得$A(2√7,y_1 )$,$B(5,y_2 )$均在第四象限内,
    反比例函数的图像在第二、第四象限内,k<0,
    在每个象限内,y随x的增大而增大,
    又2√7>5.所以 $y_1>y_2$.
    【答案】A.
    
    【答案】A
\end{answers}


\vspace{1cm}
\begin{example}
    若点A (-1,$y_1$)、B (2,$y_2$)、B (π,$y_3$)都是反比例函数$y=(k^2+1)/x$的图像上,试比较$y_1$、$y_2$、$y_3$的大小关系 \uds\uds\uds.
\end{example}

\begin{answers}
    $k^2+1$>0,所以$y=(k^2+1)/x$的函数在一、三象限内,易得$y_2>y_3>y_1$.
    【答案】$y_2>y_3>y_1$.
\end{answers}


\vspace{1cm}
\begin{example}
    已知点$A(x_1,y_1)$,$B(x_2,y_2)$是反比例函数$y=k/x (k>0)$的图象上的两点,若$x_1<0<x_2$,则有( \ \ ).\\
$A.y_1<0<y_2$	\hspace{0.5cm} $B.y_2<0<y_1$	\hspace{0.5cm} 
$C.y_1<y_2<0$	\hspace{0.5cm} $D.y_2<y_1<0$
\end{example}

\begin{answers}
    【解析】应用反比例函数的增减性,要明确已知点是否在同一支双曲线上
【答案】A
\end{answers}


\vspace{1cm}
\begin{example}
    已知反比例函数$y=(1-2m)/x$的图像上两点A ($x_1$,$y_1$),B ($x_2$,$y_2$),
    当$x_1<0<x_2$时,有$y_1<y_2$,则m的取值范围是\uds\uds.
\end{example}

\begin{answers}
    根据题意, 反比例函数y=(1-2m)/x的图像分布在第一、三象限,即1-2m>0.∴m<1/2.
    【答案】m<1/2.
\end{answers}


\vspace{1cm}
\begin{example}
    反比例函数$y=-3/x$的图像上有三点,(-2,a),(-1,b),(1,c) ,比较a,b,c大小.
\end{example}

\begin{answers}
    -3<0,在每个象限内,y随x的增大而增大.
    又点(-2,a),(-1,b)在第二象限,-2<-1,=>0<a<b.
    又点(1,c)在第四象限,=>c<0. =>c<a<b.
    【答案】c<a<b.
\end{answers}

