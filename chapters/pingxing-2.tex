%\setchapterstyle{kao}
\setchapterpreamble[u]{\margintoc}
\chapter{第二讲:平四的判定及特殊的平行四变形---矩形}
%\labch{pingxing}


上一节课,我们学习了平行四边形的基本性质,这节课我们学习:
\begin{itemize}
    \item 平行四边形的判定
    \item 临边相等的平四:菱形
    \item 角为直角的平四:矩形
    \item 临边相等且直角的平四:正方形
\end{itemize}


\section{平行四边形的判定}
\marginnote{平四的性质即判定,判定即性质!}

\begin{example}

    $$
    \begin{aligned}
        &\text{(23-24八年级下·江苏南通·期中)如图,四边形ABCD中,对角线}\\
        &\text{AC、BD相交于点O,下列条件不能判定这个四边形是平行四边形}\\
        &\text{的是(  \ \ \  )}\\
        &A.AB//DC, AD//BC \hspace{1cm}	B.AB//DC, AD=BC\\
        &C.AO=CO, BO=DO  \hspace{1cm}	 D.AB=DC, AD=BC\\
    \end{aligned}
    $$

    \includegraphics[width=0.4\textwidth]{2-exp-1.png}

\end{example}

\begin{answers}
    B
\end{answers}


\begin{example}
    1.(23-24八年级下·江苏泰州·期中)如图,在四边形ABCD中,对角线AC与BD相交于点O,下列四个选项中不能判定四边形ABCD 是平行四边形的是(  \ \    )\\
    A.∠BAD=∠BCD 	\ \ \ \ \ \ \  B.AD//BC,AB=CD\\
    C.OA=OC,OB=OD	\ \ \ D.AD//BC,AO=CO\\
    \includegraphics[width=0.35\textwidth]{2-exp-2.png}
\end{example}

\vspace{1cm}

\begin{example}
    (23-24九年级下·江苏南通·阶段练习)
    \sidenote[提示]{平四的证明问题,经常借助全等三角形}
    如图,在□ABCD中,E、F为对角线BD上两点,BE=DF.求证:四边形 ABCD 是平行四边形.\\
    \includegraphics[width=0.4\textwidth]{2-exp-3.png}
\end{example}

\vspace{1cm}

\begin{answers}
    \includegraphics[width=0.4\textwidth]{2-exp-3-ans.png}
\end{answers}


\begin{example}
    如图,四边形ABCD对角线交于点O,且O为AC中点,AE=CF,DF//BE,求证:四边ABCD是平行四边形.\\
    \\
    \includegraphics[width=0.3\textwidth]{2-exp-4.png}
\end{example}

\vspace{1cm}


\begin{example}
	如图,在平行四边形ABCD中,点M、N是对角线AC上的点,且AM=CN,DE=BF,求证:四边形MFNE是平行四边形.
    \\
    \includegraphics[width=0.4\textwidth]{2-exp-5.png}
\end{example}

\begin{answers}
    ∴∠MAF=∠NCE\\
    又∵DE=BF\\
    ∴AF=CE\\
    又∵AM=CN\\
    显然ΔAFM≌ΔCEN\\
    ∴FM=EN且∠AMF=∠CNE\\
    ∴∠FMN=∠ENM\\
    ∴四边形MFNE是平行四边形.\\
\end{answers}

\vspace{1cm}

\begin{example}
    如图,过四边形ABCD对角线的交点O作直线EF交AD、BC分别于E、F,又G、H分别为OB、OD的中点,求证:四边形EHFG为平行四边形.
    \\ \\
    \includegraphics[width=0.4\textwidth]{2-exp-6.png}
\end{example}

\vspace{1cm}

\begin{example}
    已知:如图,平行四边形ABCD内有一点E满足ED⊥AD于点D,∠EBC=∠EDC,∠ECB=45°,请找出与BE相等的一条线段,并给予证明.
    \\ \\
    \includegraphics[width=0.4\textwidth]{2-exp-7.png}
\end{example}

\vspace{1cm}

\begin{answers}
    \includegraphics[width=0.4\textwidth]{2-exp-8.png}
\end{answers}

\begin{example}
    如图,在平行四边形ABCD的各边AB,BC,CD,DA上,分别取E,F,G,H,使AE=CG,
    BF=DH,求证:四边形EFGH为平行四边形
    \\
    \includegraphics[width=0.4\textwidth]{2-exp-9.png}
\end{example}


\cleardoublepage
\section{矩形}
\begin{marginfigure}
    \includegraphics{2.2-exp-9.png}
\end{marginfigure}

\begin{definition}
    \ubi{矩形的定义}\\
    有一个角是\ubi{直角}的平行四边形叫做矩形.
\end{definition}

\begin{theorem}
    \ubi{矩形的性质}\\
    矩形是特殊的平行四边形,它具有平行四边形的所有性质,还具有自己独特的性质:
    \begin{itemize}
        \item 边的性质:对边平行且相等. 
        \item 角的性质:四个角都是\ubi{直角}.
        \item 对角线性质:对角线互相平分且\ubi{相等}.\sidenote[注意]{对角线相等的平行四边形才是矩形,如果只说对角线相等,那还可能是我们的“老朋友”等腰梯形}
        \item 对称性:矩形是中心对称图形,也是\ubi{轴对称图形}.
    \end{itemize}
\end{theorem}

直角三角形斜边上的中线等于斜边的一半.
直角三角形中,30°角所对的边等于斜边的一半.

\begin{theorem}
    \ubi{矩形的判定}
    \begin{itemize}
        \item 判定1:有一个角是直角的平行四边形是矩形.
        \item 判定2:对角线相等的平行四边形是矩形.
        \item 判定3:有三个角是直角的四边形是矩形.
    \end{itemize}
\end{theorem}


\begin{example}
    矩形具有而平行四边形不具有的性质为(  \ \   )\\
A.对角线相等      \hspace{1.7cm}                 B.对角相等\\
C.对角线互相平分    \hspace{1cm}                D.对边相等\\
\end{example}

\begin{answers}
    A
\end{answers}

\marginnote{例二在考察\ubi{特殊角}对于矩形的形状的影响。结合矩形天生具有的直角,就可以利用
勾股定理等手段进行解决}
\begin{example}
	如图,矩形ABCD沿AE折叠,使D点落在BC边上的F点处,如果∠BAF=60°,则∠DAE=\uds\\         
    \includegraphics[width=0.5\textwidth]{2.2-exp-1.png}
\end{example}

\begin{answers}
    15°
\end{answers}

\vspace{1cm}

\marginnote{绘草图的能力是考场上的重要能力之一,务必要训练自己简单快速绘图的能力,尽量不用直尺,太浪费时间。明白
图形是如何\ubi{“生长”}的,这个过程能帮助你快速了解一遍题意。这对于几何问题尤其重要。}
\begin{exercise}
    矩形ABCD中,点H为AD的中点,P为BC上任意一点,PE⊥HC交HC于点E,PF⊥BH交BH于点F,当AB,BC满足 \uds \uds 条件时,四边形PEHF是矩形
\end{exercise}


\begin{example}
    如图,在四边形ABCD中,∠ABC=∠BCD=90°,AC=BD,求证:四边形ABCD是矩形.\\
    \includegraphics[width=0.4\textwidth]{2.2-exp-2.png}
\end{example}


\begin{answers}
    \includegraphics[width=0.5\textwidth]{2.2-exp-2-ans.png}
\end{answers}

\vspace{1cm}

\begin{exercise}
    如图,已知在四边形ABCD中,AC⊥DB交于O,E、F、G、H分别是四边的中点,求证四边形EFGH是矩形.\\
    \includegraphics[width=0.5\textwidth]{2.2-exp-3.png}
\end{exercise}
\begin{answers}
    \includegraphics[width=0.5\textwidth]{2.2-exp-3-ans.png}
\end{answers}
\vspace{1cm}

\begin{example}
    如图,在平行四边形ABCD中,M是AD的中点,且MB=MC,
求证:四边形ABCD是矩形.\\
\includegraphics[width=0.4\textwidth]{2.2-exp-4.png}
\end{example}

\begin{answers}
    \includegraphics[width=0.4\textwidth]{2.2-exp-4-ans.png}
\end{answers}

\vspace{1cm}


\begin{example}
    \sidenote{联想平四中的面积相关推论,以及对角线等分面积推论,列出式子!}
    已知矩形ABCD和点P,当点P在矩形ABCD内时,试求证:$S_{△PBC}=S_{△PAC}+S_{△PCD}$\\
    \includegraphics[width=0.4\textwidth]{2.2-exp--5.png}
\end{example}

\begin{answers}
    \includegraphics[width=0.4\textwidth]{2.2-exp--5-ans.png}
\end{answers}

\vspace{1cm}

\begin{example}
    如图,将矩形ABCD沿AC翻折,使点B落在点E处,连接DE、CE,过点E作EH⊥AC,垂足为H.\\
    ⑴判断ACED是什么图形,并加以证明;\\
    ⑵若AB=8,AD=6.求DE的长;\\
    ⑶四边形ACED中,比较AE+EC与AC+EH的大小.\\
    \includegraphics[width=0.4\textwidth]{2.2-exp-6.png}
\end{example}

\begin{answers}
    \includegraphics[width=0.4\textwidth]{2.2-exp-6-ans.png}
\end{answers}

\vspace{1cm}


\cleardoublepage
\section{课后作业}


\begin{exercise}
    如图,在矩形ABCD中,点E是BC上一点,AE=AD,DF⊥AE,垂足为F.线段DF与图中的哪一条线段相等?先将你猜想出的结论填写在下面的横线上,然后再加以证明。即DF= \uds .(写出一条线段即可)\\
    \includegraphics[width=0.4\textwidth]{2.2-hw-1.png}
\end{exercise}

\begin{answers}
    \includegraphics{2.2-hw-1-ans.png}
\end{answers}


\begin{exercise}
	如图,矩形ABCD的两条对角线相交于点O,∠AOB=60°,AB=2,则矩形的对角线AC的长是( \ \ )\\
A.2  \ \ \ B.4 \ \ \ C.2√3 \ \ \  D.4√3\\
\includegraphics[width=0.4\textwidth]{2.2-hw-2.png}
\end{exercise}

\begin{answers}
    ∵∠AOB=60°,AO=BO,∴ΔAOB为等边三角形,∴AC=4
\end{answers}


\begin{exercise}
    如图,矩形ABCD中,对角线AC,BD相交于点O,AE⊥BO于E,OF⊥AD于F,已知
    OF=3cm,且BE:ED=1:3,求BD的长.\\
    \includegraphics[width=0.4\textwidth]{2.2-hw-3.png}
\end{exercise}

\begin{answers}
    \includegraphics[width=0.4\textwidth]{2.2-hw-3-ans.png}
\end{answers}

\begin{exercise}
    如图所示,在Rt⁡Δ ABC中,∠ABC=90°,将Rt⁡Δ ABC绕点C顺时针方向旋转60°得到ΔDEC点E在AC上
    ,再将Rt⁡Δ ABC沿着AB所在直线翻转180°得到ΔABF连接AD. \\
⑴ 求证:四边形AFCD是菱形;\\
⑵ 连接BE并延长交AD于G连接CG,请问:四边形ABCG是什么特殊
平行四边形?为什么?\\
\includegraphics[width=0.4\textwidth]{2.2-hw-4.png}
\end{exercise}

\begin{answers}
    \includegraphics[width=0.4\textwidth]{2.2-hw-4-ans.png}
\end{answers}