%\setchapterstyle{kao}
\setchapterpreamble[u]{\margintoc}
\chapter{第一讲:平行四边形的性质}
%\labch{pingxing}

\section{轴对称与中心对称}

上一节课,我们学习了平行四边形的基本性质,这节课我们学习:
\begin{itemize}
    \item 平行四边形的判定
    \item 临边相等的平四:菱形
    \item 角为直角的平四:矩形
    \item 临边相等且直角的平四:正方形
\end{itemize}


\section{平行四边形的判定}
平四的性质即判定,判定即性质!

\begin{example}

    $$
    \begin{aligned}
        &\text{(23-24八年级下·江苏南通·期中)如图,四边形ABCD中,对角线}\\
        &\text{AC、BD相交于点O,下列条件不能判定这个四边形是平行四边形}\\
        &\text{的是(  \ \ \  )}\\
        &A.AB//DC, AD//BC \hspace{1cm}	B.AB//DC, AD=BC\\
        &C.AO=CO, BO=DO  \hspace{1cm}	 D.AB=DC, AD=BC\\
    \end{aligned}
    $$

    \includegraphics[width=0.4\textwidth]{2-exp-1.png}

\end{example}

\begin{answers}
    B
\end{answers}


\begin{example}
    1.(23-24八年级下·江苏泰州·期中)如图,在四边形ABCD中,对角线AC与BD相交于点O,下列四个选项中不能判定四边形ABCD 是平行四边形的是(  \ \    )\\
    A.∠BAD=∠BCD, 	\ \ \ \ \  B.AD//BC,AB=CD\\
    C.OA=OC,OB=OD	\ \ \ D.AD//BC,AO=CO\\
    \includegraphics[width=0.3\textwidth]{2-exp-2.png}
\end{example}


\begin{example}
    (23-24九年级下·江苏南通·阶段练习)如图,在□ABCD中,E、F为对角线BD上两点,BE=DF.求证:四边形 是平行四边形.\\
    \includegraphics[width=0.5\textwidth]{2-exp-3.png}
\end{example}

\begin{answers}
    \includegraphics[width=0.5\textwidth]{2-exp-3-ans.png}
\end{answers}


\begin{example}
    如图,四边形ABCD对角线交于点O,且O为AC中点,AE=CF,DF∥BE,求证:四边形ABCD是平行四边形.\\
    \includegraphics[width=0.5\textwidth]{2-exp-4.png}
\end{example}


\begin{example}

\end{example}



\section{菱形}





\section{矩形}






\section{正方形}