%\setchapterstyle{kao}
\setchapterpreamble[u]{\margintoc}
\chapter{第三讲:特殊的平行四边形---菱形与正方形}
%\labch{pingxing}

\section{菱形的基本概念}

\begin{definition}
    \ubi{菱形的定义}:有一组\ubi{邻边相等}的平行四边形叫做菱形.
\end{definition}



\begin{marginfigure}
    \vspace{2cm}
    \includegraphics[width=0.7\textwidth]{3-parallel-1.png}
\end{marginfigure}

\begin{theorem}
    \ubi{菱形的性质}:菱形是特殊的平行四边形,它具有平行四边形的所有性质,还具有自己独特的性质:
    \begin{itemize}
        \item 边的性质:平四 + 临边相等(菱形独有). 
        \item 角的性质:邻角互补,对角相等(平四共有).
        \item 对角线性质:平四 + 对角线垂直且每条对角线平分一组对角(菱形独有).
        \item 对称性:平四是中心对称图形 + 菱形是轴对称图形(菱形独有).
    \end{itemize}
\end{theorem}



\begin{theorem}
    \ubi{对角线互相垂直和面积}:
    \begin{itemize}
        \item 菱形的面积等于底乘以高,等于对角线乘积的一半.
        \item 重要推论:任意四边形\ubi{对角线互相垂直},则有,\ubi{对角线乘积}\\ 
        \ubi{等于四边形面积的2倍}.
    \end{itemize}
\end{theorem}



\includegraphics[width=0.6\textwidth]{3-pllg-area.png}

$
AC×DB =(AO+OC)×(DO+OB)\\
=AO⋅DO+AO⋅OB+OC⋅DO+OC⋅OB\\
=2×(\frac{1}{2}AO⋅DO+\frac{1}{2}AO⋅OB+\frac{1}{2}OC⋅DO+\frac{1}{2}OC⋅OB)\\
=2×(S_{△ODA}+S_{△OAB}+S_{△OCD}+S_{△OCB})\\
=2×S_{\pllg ABCD}\\
$



\begin{theorem}
    \ubi{菱形的判定}:
    \begin{itemize}
        \item 一组邻边相等的平行四边形是菱形.
        \item 对角线互相垂直的平行四边形是菱形.
        \item 四边相等的四边形是菱形.
    \end{itemize}
\end{theorem}




\section{菱形相关习题}


\begin{example}
    如图1所示,菱形ABCD中,对角线AC、BD相交于点O,H为AD边中点,菱形ABCD的周长为24,则OH的长等于\uds.\\
    \includegraphics[width=0.5\textwidth]{3-exp-1.png}
\end{example}

\begin{answers}
3
\end{answers}




\begin{example}
    如图,已知菱形ABCD的对角线AC=8 cm⁡ ,BD=4 cm⁡ ,DE⊥BC于点E,则DE的长为 \uds.\\
    \includegraphics[width=0.5\textwidth]{3-exp-2.png}    
\end{example}


\begin{example}
    菱形ABCD中,∠A∶∠B=1∶5,若周长为8,则此菱形的高等于   \uds.
\end{example}

\begin{answers}
    1/2
\end{answers}

\vspace{3cm}

\begin{example}
    如图2,一活动菱形衣架中,菱形的边长均为16cm若墙上钉子间的距离AB=BC=16cm,则
    ∠1=  \uds      度.\\
    \includegraphics[width=0.5\textwidth]{3-exp-3.png}
\end{example}

\begin{answers}
120°
\end{answers}

\cleardoublepage
\begin{example}
    如图,在菱形ABCD中,AB=4a,E在BC上,BE=2a,∠BAD=120°,P点在BD上,则PE+PC的最小值为   \uds.\\
    \includegraphics[width=0.5\textwidth]{3-exp-4.png}
\end{example}


\begin{answers}
    $2\sqrt{3}a$
\end{answers}












\section{正方形的相关概念}
\begin{definition}
    \ubi{正方形的定义}:
    有一组邻边相等,并且有一个角是直角的平行四边形叫做正方形.
\end{definition}

\begin{theorem}
    \ubi{正方形的性质}:
    \begin{itemize}
        \item 边的性质:对边平行,四条边都相等.
        \item 角的性质:四个角都是直角.
        \item 对角线性质:两条对角线互相\ubi{垂直平分且相等},每条对角线平分一组对角.
        \item 对称性:正方形是中心对称图形,也是轴对称图形.
    \end{itemize}
\end{theorem}

\begin{theorem}
    \ubi{正方形的判定}:
    \begin{itemize}
        \item 有一组邻边相等的矩形是正方形.
        \item 有一个角是直角的菱形是正方形.
    \end{itemize}
\end{theorem}


\section{正方形相关习题}

\begin{example}
    如图,在正方形ABCD中,E为AB边的中点,G,F分别为AD,BC边上的点,若AG=1,BF=2,∠GEF=90°,则GF的长为 \uds     .\\
    \includegraphics[width=0.5\textwidth]{3-2exp-1.png}
\end{example}

\begin{answers}
    3
\end{answers}

\begin{example}
    如图,E是正方形ABCD对角线BD上的一点,求证:AE=CE.\\
    \includegraphics[width=0.5\textwidth]{3-2exp-2.png}
\end{example}

\begin{answers}
因为四边形ABCD是正方形\\
所以AB=BC \\
∠ABD=∠CBD\\
又BE是公共边 \\
所以ΔABE≌ΔCBE \\
所以AE=CE\\
\end{answers}


\begin{example}
    如图,已知P是正方形ABCD内的一点,且ΔABP为等边三角形,那么∠DCP=       \uds     .\\
    \includegraphics[width=0.5\textwidth]{3-2exp-3.png}
\end{example}


\begin{answers}
    15°
\end{answers}


\begin{example}
    如图,在正方形ABCD中,E为CD边上的一点,F为BC延长线上的一点,CE=CF,∠FDC=30°,求∠BEF的度数.\\
    \includegraphics[width=0.5\textwidth]{3-2exp-4.png}
\end{example}

\begin{answers}
    \includegraphics[width=0.5\textwidth]{3-2exp-4-ans.png}
\end{answers}


\cleardoublepage

\begin{example}
    如图,在正方形ABCD中,E、F分别是AB、BC的中点,求证:AM=AD.\\
    \includegraphics[width=0.5\textwidth]{3-2exp-5.png}
\end{example}


\begin{answers}
    	\includegraphics[width=0.5\textwidth]{3-2exp-5-ans.png}
\end{answers}


\begin{example}
    如图,A在线段BG上,ABCD和DEFG都是正方形,面积分别为7 $cm^2$ ⁡和11 $cm^2$⁡,则ΔCDE的面积为\uds.\\
    \includegraphics[width=0.7\textwidth]{3-2exp-6.png}           
\end{example}

\begin{answers}
    \includegraphics[width=0.5\textwidth]{3-2exp-6-ans.png}
\end{answers}


\begin{example}
    若正方形ABCD的边长为4,E为BC边上一点,BE=3,M为线段AE上一点,射线BM交正方形的一边于点F,且BF=AE,则BM的长为 \uds.\\
\end{example}

\begin{answers}
    \includegraphics[width=0.5\textwidth]{3-2exp-7-ans.png}
\end{answers}

\vspace{3cm}


\cleardoublepage
\section{三角形的中位线}
\begin{definition}
    \ubi{三角形的中位线}
    \begin{itemize}
        \item \ubi{中位线}:连结三角形两边的中点所得的线段叫做三角形的中位线.
    \end{itemize}
\end{definition}

中位线两种做法:
\begin{enumerate}
    \item 过三角形一边中点作平行线与另一边交点所得线段
    \item 连接三角形两边的中点所得线段
\end{enumerate}


\begin{theorem}
    \ubi{三角形的中位线性质}:
    \begin{itemize}
        \item 三角形的中位线平行第三边且长度等于第三边的一半.
    \end{itemize}
\end{theorem}   

\section{三角形的中位线相关习题}

\begin{example}
	如图,在四边形ABCD中,AB≠CD,E、F、G、H分别是AB、BD、CD、AC的中点,要使四边形EFGH是菱形,四边形ABCD还满足的一个条件是 \uds ,并说明理由.\\
    \includegraphics[width=0.5\textwidth]{3-2exp-8.png}
\end{example}

\begin{answers}
    AD=BC
\end{answers}


\begin{example}
    如图,四边形ABCD中,E,F分别是边AB,CD的中点,则AD,BC和EF的关系是( \ \  )\\
    A.AD+BC>2EF     \ \ \         B.AD+BC≥2EF\\
    C.AD+BC<2EF     \ \ \         D.AD+BC≤2EF\\
    \includegraphics[width=0.5\textwidth]{3-2exp-9.png}
\end{example}

\begin{answers}
    B
    \includegraphics[width=0.5\textwidth]{3-2exp-9-ans.png}
\end{answers}


\begin{example}
    如图,四边形ABCD中,AB=CD,E,F,G,H分别是AD,BC,BD,AC的中点,求证:EF,GH相互垂直平分.\\
    \includegraphics[width=0.5\textwidth]{3-2exp-10.png}
\end{example}

\begin{answers}
    \includegraphics[width=0.5\textwidth]{3-2exp-10-ans.png}
\end{answers}


\begin{example}
    如图,在四边形ABCD中,M、N分别为AD、BC的中点,BD=AC,BD和AC相交于点O,MN分别与AC、BD相交于E、F,求证:OE=OF.\\
    \includegraphics[width=0.5\textwidth]{3-2exp-11.png}
\end{example}

\begin{answers}
    \includegraphics[width=0.5\textwidth]{3-2exp-11-ans.png}
\end{answers}


\begin{example}
    如图,ΔABC中,AD是∠BAC的平分线,CE⊥AD于E,M为BC的中点,AB=14cm,AC=10cm,则ME的长为 \uds.\\
    \includegraphics[width=0.5\textwidth]{3-2exp-12.png}
\end{example}

\begin{answers}
    \includegraphics[width=0.5\textwidth]{3-2exp-12-ans.png}
\end{answers}