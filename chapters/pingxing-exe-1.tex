%\setchapterstyle{kao}
\setchapterpreamble[u]{}
\chapter{课堂小测1:平四的概念与性质}
\date{\today}
%\labch{pingxing}

姓名:\uds\uds   \ \ \ \ 成绩:\uds\uds \\
\begin{exercise}
    $$\begin{aligned}&\text{如图,在平行四边形}ABCD\text{中,}AC,BD\text{相交于点}O,AB=10,AD=8,AC\bot BC,\text{则}OB=\uds\\\end{aligned}$$
    \includegraphics[width=0.5\textwidth]{1-exam-1.png}
\end{exercise}

\begin{answers}
 $\sqrt{73}$
\end{answers}



\begin{exercise}
    $$\begin{aligned}&\text{已知平行四边形一边长为10,一条对角线长为6,则它的另一条对角线}a\text{的取值范围为(\ \ )}.\\
    &A. 4<a<16  \hspace{1cm}  B. 14<a<26  \hspace{1cm}  C. 12<a<20  \hspace{1cm} D. \text{以上答案都不正确}
    \end{aligned}$$
\end{exercise}

\begin{answers}
B
\end{answers}

\vspace{2cm}

\begin{exercise}
    $$\begin{aligned}&\text{如图,}\boldsymbol{EF}\text{过平行四边形}
        \boldsymbol{ABCD}\text{对角线的交点}\boldsymbol{O}\text{,交}\boldsymbol{AD}\text{于}
        \boldsymbol{E}\text{,交}\boldsymbol{BC}\text{于}\boldsymbol{F}\text{,若平行四边形}\boldsymbol{ABCD}
        \text{的}\\&\text{周长为18,}\boldsymbol{OE}=\boldsymbol{1}.\boldsymbol{5}\text{,则四边形}
        \boldsymbol{EFCD}\text{的周长为}\uds.\end{aligned}$$
    \includegraphics[width=0.5\textwidth]{1-exam-2.png}
\end{exercise}

\begin{answers}
\includegraphics{1-exam-2-ans.png}
\includegraphics{1-exam-2-ans2.png}
\end{answers}


\vspace{3cm}

\begin{exercise}
    $$\begin{aligned}&\text{如图所示,}P\text{是平行四边形}ABCD\text{内一点,且}S_{\Delta PAB}=5\mathrm{,}S_{\Delta PAD}=2\text{,则阴影部分的面积}\\&\text{为}\uds.\end{aligned}$$
    \includegraphics[width=0.5\textwidth]{1-exam-3.png}
\end{exercise}

\begin{answers}
    \includegraphics{1-exam-3-ans.png}
\end{answers}


\begin{exercise}
$$\begin{aligned}
&\text{如图,在平行四边形ABCD中,点E在BC边上,且AD=DE,F为线段DE上一点,且∠AFE=∠B.}\\
&\text{(1)求证:∠AFD=∠ECD;}\\
&\text{(2)求证:△AFD \cong △DCE;}\\
\end{aligned}$$
\includegraphics[width=0.5\textwidth]{1-exam-4.png}
\end{exercise}

\begin{answers}
    (1)证明: ∵四边形ABCD是平行四边形,\\
∴∠B+∠ECD=180°;\\
∵∠AFE=∠B,∠AFE+∠AFD=180°,\\
∴∠AFD=∠ECD;\\
(2)证明:∵四边形ABCD是平行四边形\\
∴AD∥BC,\\
∴∠ADF=∠DEC,\\
由(1)知:∠AFD=∠ECD,\\
∵AD=DE,\\
∴△AFD≌△DCE("AAS");\\
(3)解:∵△AFD≌△DCE,\\
∴AF=CD;\\
∵CD=DF,\\
∴AF=DF,\\
∵∠B=∠AFE="60" °,\\
∴∠ADF=∠DAF=30°;\\
∵AD=DE,\\
∴∠DAE=∠DEA=75°,\\
∴∠EAG=45°;\\
∵四边形ABCD是平行四边形\\
∴AD∥BC,\\
∴∠DAE=∠AEB,\\
∴∠DEA=∠AEB;\\
∵∠AFE=∠B,AE=AE,\\
∴△ABE≌△AFE,\\
∴EF=BE=2;\\
如图,过E作EG⊥AF于G,\\
则∠AEG=∠EAG=45°,\\
∴AE=EG,∠GEF=∠AEF-∠AEG=30°;\\
在"Rt"△EGF中,FG=1/2 EF=1,由勾股定理得EG=$√(EF^2-FG^2 )=√3$,\\
在"Rt"△EGA中,AG=EG=√3,由勾股定理得AE=$√(AG^2+EG^2 )=√6$.\\
\includegraphics{1-exam-4-ans.png}
\end{answers}

\vspace{2cm}


\begin{exercise}
    现有如图2的铁片,其形状是一个大的平行四边形在一角剪去一个小的平行四边形,工人师傅想用一条直线将其分割成面积相等的两部分,请你帮助师傅设计三种不同的分割方案
    \\ \\
    \includegraphics[width=0.3\textwidth]{1-exam-5.png}
    \includegraphics[width=0.3\textwidth]{1-exam-5.png}
    \includegraphics[width=0.3\textwidth]{1-exam-5.png}
\end{exercise}