%\setchapterstyle{kao}
\setchapterpreamble[u]{}
\chapter{课堂小测2:矩形的相关性质与证明}
\date{\today}
%\labch{pingxing}

姓名:\uds\uds   \ \ \ \ 成绩:\uds\uds \\
\begin{exercise}
    \sidenote{注意矩形性质中对角线相等的运用}
    如图,在中,点是边上的一个动点,过点作直线,若交的平分线于点,交的外角平分线于点\\
    (1)求证:DE=DF\\
    (2)当点运动到何处时,四边形为矩形?请说明理由。\\
    \includegraphics[width=0.5\textwidth]{2-exam-1.png}
\end{exercise}

\begin{answers}
    ⑴证明:ED=DC,DF=DC\\
    ⑵当D为AC的中点时,四边形AECF为矩形\\
\end{answers}

\stu{4}

\begin{exercise}
    \sidenote[提示]{同上,注意矩形的独特性质:对角线相等}
	已知,如图,矩形ABCD中,CE⊥BD于E,AF平分∠BAD交EC于F,求证:CF=BD.\\
    \includegraphics[width=0.5\textwidth]{2-exam-2.png}
\end{exercise}

\begin{answers}
    \includegraphics{2-exam-2-ans.png}
\end{answers}

\stu{3}

\begin{exercise}
    \sidenote[提示]{(天生有直角)矩形--直角--勾股--面积,经常作为解题combo!}
    如图,点E是矩形ABCD的对角线BD上的一点,且BE=BC,AB=3,BC=4,点P为直线EC上的一点,且PQ⊥BC于点Q,PR⊥BD于点R.
    \begin{enumerate}
        \item 如图1,当点P为线段EC中点时,易证:PR+PQ=12/5(不需证明).
        \item 如图2,当点P为线段EC上的任意一点(不与点E、点C重合)时,其它条件不变,则(1) 中的结论是否仍然成立?若成立,请给予证明;若不成立,请说明理由.
        \item 如图3,当点P为线段EC延长线上的任意一点时,其它条件不变,则PR与PQ之间又具有怎样的数量关系?请直接写出你的猜想
    \end{enumerate}

\includegraphics[width=1.2\textwidth]{2-exam-3.png}
\end{exercise}

\begin{answers}
    \includegraphics{2-exam-3-ans.png}
\end{answers}