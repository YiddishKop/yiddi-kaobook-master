%\setchapterstyle{kao}
\setchapterpreamble[u]{}
\chapter{课堂小测3:矩形的相关性质与证明}
\date{\today}
%\labch{pingxing}

姓名:\uds\uds   \ \ \ \ 成绩:\uds\uds \\
\begin{exercise}
    如图,矩形$ABCD$中,$AC,BD$相交于点$O$, $AE$平分$\angle BAD$交$BC$于$E$ ,若$\angle CAE=15^{\circ}$,求
$\angle BOE=\uds$
    \\
    \includegraphics[width=0.5\textwidth]{3-exe-1.png}
\end{exercise}


\begin{answers}

\end{answers}


\stu{4}



\begin{exercise}
	如图,平行四边形ABCD中,E是BC的中点,DE、AB的延长线交于点F,连接AE、CF.求证:$S_{ΔABE}=S_{ΔEFC}$.\\
    \includegraphics[width=0.5\textwidth]{3-exe-2.png}
\end{exercise}

\begin{answers}
    \includegraphics[width=0.5\textwidth]{3-exe-2-ans.png}
\end{answers}

\stu{3}

\cleardoublepage
\begin{exercise}
    如图,ΔACD、ΔABE、ΔBCF均为直线BC同侧的等边三角形.当AB≠AC时,\\
    (1)求证:△BEF $\cong$ △BAC\\
    (2)求证:四边形ADFE为平行四边形\\
    (3)若AB=3,AC=2,∠BAC=120°,求四边形AEFD的面积.\\
    \includegraphics[width=0.5\textwidth]{3-exe-3.png}
\end{exercise}

\begin{answers}
(1)BA=BE  , BC=BF  , ∠CBF=∠ABE=60°  , \\
∴∠CBA=∠FBE∴△BEF≌△BAC,\\

(2)∵ΔABE、ΔBCF为等边三角形,\\
∴AB=BE=AE,BC=CF=FB,∠ABE=∠CBF=60°.\\
∴∠FBE=∠CBA.\\
∴ΔFBE≌ΔCBA.\\
∴EF=AC.\\
又∵ΔADC为等边三角形,\\
∴CD=AD=AC.\\
∴EF=AD.\\
同理可得AE=DF.\\
∴四边形AEFD是平行四边形.\\

(3)∠DAE=120°,AE=AB=3  , AC=AD=2,过D点作出AE边上的高,就可求得面积为3√3.\\
\end{answers}
